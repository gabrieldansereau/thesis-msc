Beta diversity, defined as the variation in species composition among
sites in a geographic region of interest \citep{hastings1991}, is an
essential measure to describe the organization of biodiversity though
space. Total beta diversity within a community can be partitioned into
local contributions to beta diversity (LCBD) \citep{hastings1991}, which
allows the identification of sites with exceptional species composition,
hence unique biodiversity. Such a method, focusing on specific sites, is
useful for both community ecology and conservation biology, as it
highlights areas that are most important for their research or
conservation values. However, the use of LCBD indices is currently
limited in two ways. First, LBCD indices are typically used on data
collected over local or regional scales with relatively few sites, for
example on fish communities at intervals along a river or stream
\citep{hastings1991}. Second, LCBD calculation methods require complete
information on community composition, such as a community composition
matrix \(Y\); thus, they are inappropriate for partially sampled sites
(e.g.~where data for some species is missing), let alone for unsampled
ones. Accordingly, the method is of limited use to identify areas with
exceptional biodiversity in regions with sparse sampling. However,
predictive approaches are increasingly common given the recent
development of computational methods, which often uncover novel
ecological insights from existing data \citep{hastings1991}, including
in unsampled or lesser-known locations, as well as larger spatial
scales. Here, we examine whether the LCBD method can assess ecological
uniqueness over broad and continuous scales based on predictions of
species distributions and evaluate whether this reveals novel ecological
insights regarding the identification of exceptional biodiversity areas.
